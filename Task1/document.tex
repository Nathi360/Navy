\documentclass[10pt,a4paper]{article}
\usepackage[latin1]{inputenc}

\author{Navy Team}


\title{NavUP : Software Requiremnts Specification}
\begin{document}
	\maketitle

	\section{Introduction}

		\subsection{Purpose}

		The goal of this Software Requirements Specification is to ascertain the
		concrete necessities that the proposed software, NavUP, should fulfil. This
		software, amongst other tasks, will act primarily as a navigation tool. For
		this reason we believe the two main interest groups are:

		\begin{itemize}
			\item Students who are search for a particular building on campus, or are
						investigating the many historically relevant buildings on campus.

			\item Guests to the university that require assistance finding locations.
		\end{itemize}


		\subsection{Scope}

			The system proposed will be called NavUP. Put briefly, the product will
			off on-campus navigation support to students, lecturers, and visitors to
			the university.

		\subsection{Definitions, Acronyms and Abbreviations}

		\subsection{References}

		\subsection{Overview}

			This SRS document is structured...

	\section{Overall Description}

		The NavUP Navigation system

		\subsection{Product Perspective}

			\subsubsection{System Interfaces}

			\subsubsection{User Interfaces}

				There will be three main mediums through which prospective users can
				interact with the proposed system. These are:

				\paragraph{Web Interface}

					Users will be able to access the software through an intuitive Web
					Interface, and hence through any device that has an up-to-date browser.

				\paragraph{Android Application}

					An Android application will allow the users of Android smartphones
					and tablets to interact with the system with a native application on
					their device.

				\paragraph{iOS Application}

					An iOS application, similarly to the Android app, will enable mobile
					Apple devices, such as iPhones or iPads, to connect to the software
					service.

			\subsubsection{Hardware Interfaces}

				As described above, we intend to publish to three platforms.

			\subsubsection{Software Interfaces}



			\subsubsection{Communication Interfaces}

			\subsubsection{Memory}

				Most information regarding the system will reside in a central
				database. This database will contain, amongst other things:

				\begin{itemize}
					\item User information
					\item Locations
					\item Events
					\item Route caching
				\end{itemize}

				For network efficiency, it will make sense to keep a portion of this
				data locally on the device in question. Similarly the browser may store
				a small fraction of this data (most likely user data) to reduce data
				usage, server strain and lookup time.

			\subsubsection{Operations}



			\subsubsection{Site Adaptation Requirements}

		\subsection{Product Functions}

			\begin{itemize}
				\item Successfully identify a user's current location, and plan an
							accurate route to another location on campus.
			\end{itemize}

		\subsection{User Characteristics}

			For the average user of the system we should not assume any more technical
			skills than that required to operate simple smartphone apps. The propsed
			software should immitate, as far as possible, the paradigms and idealogies
			that previous mobile and browser applications have established. We can
			expect that students will have encountered navigational systems such as
			Google Maps, and should thus try to meet the expectations that users have
			developed in using these application.

		\subsection{Constraints}

			\begin{itemize}

				\item We do not need to account for navigation off-campus
				\item We can only accurately navigate within sufficient WiFi coverage.

			\end{itemize}

		\subsection{Assumptions and Dependencies}



	\section{Specific Requirements}

		\subsection{External Interface Requirements}

			The system will need to effectively interface with a device's WiFi and GPS
			hardware.

		\subsection{Functional Requirements}

			The system will

		\subsection{Performance Requirements}

			The system will need to operate within reasonable levels of speed and
			responsiveness to be useful to the end-user. For this reason we will need
			to consider a path finding algorithm that is suited to both the need for
			accuracy and the time needed to calculate such a path.

			\medskip

			One strategy that we might use is to \textsl{precompute portions of the
			path-finding}. We may already know the best route between two shorter
			paths, and hence the search space of the path finding algorithm can be
			greatly reduced when composed of only two or three path ``legs'' to choose
			from.

			\medskip

			Further, since the largest number of users is expected to be mobile users,
			criteria such as battery usage and processing power will be of
			great concern. To this end, we can think about caching the data that is
			most important and most frequently used for the user on their devices.
			This will make the system feel more responsive, and will reduce overall
			strain on the system.

			\medskip

			We can also think about caching on the system itself. If two students
			request a path to the same location, and are within 20 meters of one
			another, we can consider that as two equal requests and compute this once.
			The system can keep a log of requests within storage, for a limited period
			of time. This will save massive amounts of recomputation.

			\medskip

			In terms of accuracy, the software will need provide navigation between
			individual lecture halls. This means that an accuracy with less than a 5
			meter error is essential for insuring that lecture-hall resolution can
			still work reliably.

		\subsection{Design Constraints}

			The success and performance of the proposed system is contingent on the
			following citeria which cannot be altered:

			\paragraph{Accuracy Is Limited By Device}

				Due to the fact that navigation is dependednt entirely on WiFi signal
				strength, the accuracy of the navigation can only ever be as proficient
				as the sensitivity of the WiFi hardware. This is a fundamental constaint,
				and cannot be worked around in the current vision of the proposed
				software.

			\paragraph{Processing Capability}

				Mobile devices vary significantly in terms of processing capability,
				particularly so in Android devices. In order to make the system
				available to the largest possible audience, we must build to the lowest
				common denominator. With this in mind, we must be confident that the
				application will run smoothly on a device that may only have a single
				500 Mhz processor. The only way to achieve this is to make the client
				application as light as possible, and only do essential processing on
				device itself.

				\paragraph{Navigation is Limited By Source Maps}

				The system's ability to navigate is directly proportional to the accuracy
				of the maps which we use. If the university has been poorly mapped in
				certain areas, navigation may be inaccurate or impossible altogether.

		\subsection{Software System Attributes}

			Another concern is the security of users' data. To prevent a compromise of
			private data we must follow the best practises as established by industry
			leaders, such as:

			\begin{itemize}
				\item Not storing plaintext passwords
				\item Not storing identifying datapoints
				\item Encrypting critical data communication
				\item Maintaining anonymity as far as possible
			\end{itemize}

			This system will be available at all times, and a maintainer or
			administrator will only have to interact wFith the system when
			\textsl{CRUD}-ing locations and events, or to observe usage statistics
			through the analytics dashboard.

			\medskip

			Value may also be extended by integrating third-party services. One such
			addition would be integration with \textsl{Google Calendar}. From Google
			Calendar we would be able to get timed appointments at specific locations,
			and would therefore be able to find paths. Interoperability with the UP
			network services would equally be automatically load lecture times for
			both lecturers and students.

		\subsection{Other Requirements}

\end{document}
