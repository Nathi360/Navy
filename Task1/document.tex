\documentclass[10pt,a4paper]{article}
\usepackage[latin1]{inputenc}

\author{Navy Team}


\title{NavUP : Software Requiremnts Specification}
\begin{document}
	\maketitle

	\section{Introduction}

		\subsection{Purpose}

		The goal of this Software Requirements Specification is to ascertain the
		concrete necessities that the proposed software, NavUP, should fulfil. This
		software, amongst other tasks, will act primarily as a navigation tool. For
		this reason we believe the two main interest groups are:

		\begin{itemize}
			\item Students who are search for a particular building on campus, or are
						investigating the many historically relevant buildings on campus.

			\item Guests to the university that require assistance finding locations.
		\end{itemize}


		\subsection{Scope}



		\subsection{Definitions, Acronyms and Abbreviations}

		\subsection{References}

		\subsection{Overview}

	\section{Overall Description}

		The NavUP Navigation system

		\subsection{Product Perspective}

			\subsubsection{System Interfaces}

			\subsubsection{User Interfaces}

				There will be three main mediums through which prospective users can
				interact with the proposed system. These are:

				\paragraph{Web Interface}

					Users will be able to access the software through an intuitive Web
					Interface, and hence through any device that has an up-to-date browser.

				\paragraph{Android Application}

					An Android application will allow the users of Android smartphones
					and tablets to interact with the system with a native application on
					their device.

				\paragraph{iOS Application}

					An iOS application, similarly to the Android app, will enable mobile
					Apple devices, such as iPhones or iPads, to connect to the software
					service.

			\subsubsection{Hardware Interfaces}

				As described above, we intend to publish to three platforms.

			\subsubsection{Software Interfaces}



			\subsubsection{Communication Interfaces}

			\subsubsection{Memory}

				Most information regarding the system will reside in a central
				database. This database will contain, amongst other things:

				\begin{itemize}
					\item User information
					\item Location
				\end{itemize}

				For network efficiency, . Similarly the browser may store a
				small fraction of this data (most likely user data) to reduce data usage.

			\subsubsection{Operations}



			\subsubsection{Site Adaptation Requirements}

		\subsection{Product Functions}

			\begin{itemize}
				\item Successfully identify a user's current location, and plan an
							accurate route to another location on campus.
			\end{itemize}

		\subsection{User Characteristics}

			For the average user of the system we should not assume any more technical
			skills than that required to operate simple smartphone apps. The propsed
			software should immitate, as far as possible, the paradigms and idealogies
			that previous mobile and browser applications have established. We can
			expect that students will have encountered navigational systems such as
			Google Maps, and should thus try to meet the expectations that users have
			developed in using these application.

		\subsection{Constraints}

			\begin{itemize}
				\item We do not need to account for navigation off-campus
				\item We can only accurately navigate within sufficient WiFi coverage.

			\end{itemize}

		\subsection{Assumptions and Dependencies}



	\section{Specific Requirements}

		\subsection{External Interface Requirements}

		\subsection{Functional Requirements}

		\subsection{Performance Requirements}

			The system .

		\subsection{Software System Attributes}

		\subsection{Other Requirements}

\end{document}
